\documentclass[14pt]{article}
\title{CSE 300 Online 1}
\author{1705048}
\date{\today}
\usepackage{xcolor}
\usepackage[utf8]{inputenc}
\usepackage[T1]{fontenc}
\usepackage{enumerate}
\usepackage{float}
\usepackage{multicol}
\usepackage{multirow}
\begin{document}
\maketitle
\tableofcontents
\pagebreak
\section{Introduction}
This is first online practice on \LaTeX\ .\\
We will do some styling in section 2.
\section{Styling}
    This is \textbf{\textcolor{red}{bold}}. This is \textit{italic} \& so on.
\section*{Use of Special Characters}
You can use special characters by preceding with \. Another package can be used named as verbatim. Example:\~{}$!$\$\#$\backslash$$@$$*$\^{}
\section{List}
There are 3 kinds of list.
\begin{enumerate}[i]
    \item Unordered
    \item Ordered
    \item Description
\end{enumerate}
\pagebreak
Let’s see an example of lists:
\begin{description}
    \item[CSE 311] Data Communication 1
    \begin{itemize}
        \item Fourier
        \item[] ADC
    \end{itemize}
    \item[CSE 305] Computer Architecture
    \begin{enumerate}
        \item MIPS
        \begin{enumerate}
        \item commands
        \item pipeline
        \end{enumerate}
        \item Shared Memory
    \end{enumerate}
    \item[CSE 309] Compiler
\end{description}
\section{Table}
For understanding the use of multi-row and multi-column command see the
table 1
\begin{table}[H]
    \centering
    \begin{tabular}{|c|c|c|c|}
       \hline
       \multirow{2}{*}{Dollar}  & Rupee & taka & Euro \\
	  \cline{2-4}
       & \multicolumn{2}{|c|}{\multirow{3}{*}{Dhaka}} & DU\\
	  \cline{4-4}
	  City & \multicolumn{2}{|c}{} & BUET\\
	  \cline{1-1}
	  \cline{4-4}
       Food & \multicolumn{2}{|c|}{} & MOSQUE\\
      \hline
    \end{tabular}
    \caption{  A table for online test}
\end{table}
\end{document}