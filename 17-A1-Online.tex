\documentclass[11pt]{article}
\usepackage{float}
\parindent 0px
\usepackage[utf8]{inputenc}
\usepackage{biblatex}
\addbibresource{ref.bib}
\usepackage{color}
\usepackage{xcolor}
\usepackage{graphicx}
\usepackage{multicol}
\usepackage{multirow}
\usepackage[T1]{fontenc}
\usepackage{enumerate}
\usepackage{float}
\usepackage{amsmath,mathtools,amssymb,amsfonts}
\usepackage{physics}
\usepackage{hyperref}
\usepackage[margin=1in]{geometry}
\title{CSE 300: Online Assignment}
\author{Md Shamsuzzoha Bayzid\textsuperscript{1,*}, Mahjabin Nahar \textsuperscript{1,†}, Md Shariful Islam \\  Bhuyan \textsuperscript{1,†}, and Md Saidur Rahman \textsuperscript{1,†}
\\
\textsuperscript{1}Department of Computer Science and Engineering \\ Bangladesh University of Engineering and Technology 
\\
\textsuperscript{*}Corresponding author: shams bayzid@cse.buet.ac.bd}
\date{April 07, 2021}
\begin{document}
\maketitle
\section{Introduction}
This assignment has been designed to assess the preparation of the students in writing scientific articles using \LaTeX. Different components, that are frequently used in scientific manuscripts, have been covered in this assignment.
\subsection{Tables}
We wish to place Table 1 right here.
\begin{table}[H]
    \centering
    \caption{ \textbf{Optimization scores for Method-1 and Method-2 on different datasets covering various model conditions.} We show average scores of two optimization criteria for various model conditions.}
    \vspace{1cm}
    \begin{tabular}{|c|c c|c c|c c|}
       \hline
       \multicolumn{3}{|c|}{Simulation Condition} & \multicolumn{4}{|c|}{Optimization Score}\\
       \hline
       Dataset & Complexity & Model & \multicolumn{2}{|c|}{Optimization Score 1} & \multicolumn{2}{|c|}{Optimization Score 2} \\
      \cline{4-7}
       & & condition & Method-1 & Method-2 & Method-1 & Method-2\\
      \hline
      \hline
        \multirow{4}{*}{D1} & \multirow{2}{*}{Easy} & M1 & 7,425.55  & 770.00 & 929.55 & 10\\
     & & M2 & 7,657.00 & 9,179.00 & 716.15 & 20\\
     \cline{2-7}
     & \multirow{2}{*}{Hard} & M3 & 54.00 & 9,007.15 & 3,759.00 & 30 \\
     & & M4 & 74.00 & 5567.15 & 99.00 & 25 \\
      \hline
      \hline
     \multirow{3}{*}{D2} & \multirow{3}{*}{Moderate} & M1 & 34.00 & 273.00 & 321.60 & 34\\
     & & M2 & \multicolumn{2}{|c|}{Not Applicable} & 16.00 & 11\\
     & & M3 &  657.00 & 179.60 & 716.00 & 19\\
      \hline
    \end{tabular}
\end{table}
\pagebreak
\subsection{Figure}
We intend to put Figure 1 at the top of a page.
\begin{figure}[t]
	\centering
	\includegraphics[width=5cm,height=5cm,scale=0.4]{17-CSE300_online-Figure3.pdf}
	\caption{\textbf{Nearest Neighbor Interchange (NNI) move on an internal edge.} (a) A species tree ST, and (b)-(c) the neighbors of ST resulting from one NNI move on edge e = (u1, u2). A, B, C, and D are the sets of taxa in the four subtrees around edge e.}
\end{figure}
\subsection{Equations}
Let n1|n2|n3 be a tripartition defined on an internal node u of a binary tree T. The number of tripartitions mapped to u is given by Eqn. 1.
\begin{equation}
\begin{aligned}
    \mathcal{N Q} (n1,n2,n3) &= 
   \binom{n1}{2} \binom{n2}{1} \binom{n3}{1} + \binom{n2}{2} \binom{n1}{1} \binom{n3}{1} + \binom{n3}{2} \binom{n1}{1} \binom{n2}{1}\\
   &=\frac{n_1n_2n_3(n_1 + n_2  +n_3 - 3)}{2}
\end{aligned}
\end{equation}
 \section{Conclusions}
 The major objectives of this assignment are listed below (please do not ignore the font sizes).
 \begin{itemize}
     \item \begin{huge} To assess the ability of the students in preparing manuscripts in  \LaTeX. \end{huge}
     \pagebreak
     \item \begin{Large} To see if the students have adequately practiced different aspects of writing in  \LaTeX . \end{Large}
     \item \begin{large} To see if the students can add various basic components (e.g., tables, figures, equations) to a 
     \LaTeX \space manuscript. \end{large}
      \item \begin{normalsize} To see if the students can leverage the available materials (both offline and online) to do something which has not explicitly been taught in the class. \end{normalsize}
 \end{itemize}
\printbibliography
\end{document}