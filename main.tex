\documentclass[14pt]{article}
\title{CSE 300 Online 1}
\author{1705048}
\date{\today}
\usepackage{xcolor}
\usepackage[utf8]{inputenc}
\usepackage[T1]{fontenc}
\usepackage{enumerate}
\usepackage{float}
\usepackage{multicol}
\usepackage{multirow}
\usepackage{graphicx}
\usepackage{amsmath,mathtools,amssymb,amsfonts}
\usepackage{physics}
\usepackage{biblatex}
\addbibresource{ref.bib}
\newcommand{\comb}[2]{{}_{#1}\mathrm{C}_{#2}}
\newcommand*{\MyComb}[2]{{}^{#1}C_{#2}}%
\begin{document}
\maketitle
\tableofcontents
\pagebreak
\section{Introduction}
\textbf{Google Scholar} is a wonderful search engine for finding research articles \cite{dirac}.
\begin{figure}[h]
	\centering
	\includegraphics[width=0.5cm,height=0.5cm,scale=0.4]{gs_icon.png}
	\includegraphics[width=3cm,height=0.5cm,scale=0.4]{gs_page.png}
	
	(a) Icon   (b) Banner
	
	\includegraphics[width=4cm,height=4cm,scale=0.4]{plag.jpg}
	\label{fig:1}
	\caption{Guidance towards research}
\end{figure}
\section{Equations}
Euler’s formula is one of the most important equations in mathematics. It establishes a relationship between trigonometric function and complex exponential function. The equation is as follows.
\begin{equation}
e^{i\theta}=\cos\theta + i\sin\theta
\end{equation}\par
If we put $\theta$ = $\frac{\pi}{2}$ in equation 1, we get the following:
$$\begin{aligned}
e^{i{\frac{\pi}{2}}}&=\cos\frac{\pi}{2}+i\sin\frac{\pi}{2} \\
&=0+i.1 \\
&=i
\end{aligned}$$\par
If we put $\theta$ = $\pi$, we get $e^{i\pi}+1 = 0$ which is termed as \textit{Euler’s Identity} \cite{einstein}.
\subsection{Equation Examples}
\begin{equation}
\MyComb{n}{r} = \binom{n}{r} = \frac{n!}{r!(n-r)!} 
\end{equation}
\pagebreak
% $$\left(\frac{a}{b}\right)$$
$$
\begin{array}{cc}
  F_c(x,y)= & 
    \begin{cases}
     \pdv[2]{x^{3}y^{x}}{x}+ \pdv[2]{\Gamma(x)\log(\tan y)}{x}{y} & \text{if x,y are real numbers} \\
      \lim_{z \to e^{x^{2y}}} \sqrt{z+\frac{1}{\sqrt{z+\frac{1}{\sqrt{z+...}}}}}  & \text{otherwise}
    \end{cases}
\end{array}
$$
% $$F_c(x,y)=\left\{hi\right.$$
\section*{References}
\printbibliography
\end{document}